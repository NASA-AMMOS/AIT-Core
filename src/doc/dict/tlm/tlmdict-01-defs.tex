%=======================================================================
% JPL Project Document LaTeX Template
%
% Template Author: Laura Alisic Jewell (23-Nov-2015)
%
% Copyright: 2015 California Institute of Technology.  United States
% Government sponsorship acknowledged.  ALL RIGHTS RESERVED.
%=======================================================================

\newcommand{\jpldoc}{D-81822}

\newcommand{\mission}{OCO-3}
\newcommand{\missionfull}{Orbiting Carbon Observatory-3}

\newcommand{\docname}{Telemetry Dictionary}
\newcommand{\docdate}{\today}

\newcommand{\revision}{Rev B}

\newcommand{\prepareAname}{Alan Mazer}
\newcommand{\prepareArole}{\mission \, Flight Software Lead}

\newcommand{\prepareBname}{Cecilia Cheng}
\newcommand{\prepareBrole}{\mission \, Mission System Manager}

\newcommand{\approveAname}{Matthew Bennett}
\newcommand{\approveArole}{\mission \, Project System Engineer}

\newcommand{\approveBname}{Stephen Greenberg}
\newcommand{\approveBrole}{\mission \, Mission Assurance Manager}

\newcommand{\projectlibloc}{https://alpha-lib.jpl.nasa.gov/docushare/dsweb/View/Library-228}

\newcommand{\nasajplbottom}{\begingroup
\par\vspace*{\fill}
National Aeronautics and\\
Space Administration\\

% JPL logo
\begin{figure}[h]
    \includegraphics[width=0.3\textwidth]{figures/jpl-logo}
\end{figure}

4800 Oak Grove Drive\\
Pasadena, California 91109-8099\\
\endgroup}

\newcommand{\tlmname}[1]{\texttt{#1}}
\newcommand{\tlmarg}[1]{\texttt{#1}}
\newcommand{\argenum}[2]{\texttt{#1: #2}}

\newenvironment{tlmusage}[1]
{
  \subsubsection*{Usage}
  \hangindent=0.7cm \tlmname{#1} \enspace
}
{
}

\newenvironment{argdesc}
{
  \subsubsection*{Where}
  \vspace{-0.5cm}
  \renewcommand{\arraystretch}{1.5}
  \table[h]
    \center
      \tabularx{\textwidth}{lX}
}
{
       \endtabularx
    \endcenter
    \vspace{-1cm}
  \endtable
}

\newenvironment{tlmdetails}
{
\renewcommand{\arraystretch}{1.5}
%  \table[htp]
%    \center
\begin{longtable}{|p{0.05\textwidth}|p{0.3\textwidth}|p{0.5\textwidth}|p{0.15\textwidth}|}
%      \tabularx{\textwidth}{|l|l|X|l|}
\hline
\textbf{Byte}  & \textbf{Field} & \textbf{Description} & \textbf{Type}\\
\hline
}
{
%      \endtabularx
\end{longtable}
%    \endcenter
%    \vspace{-1cm}
%  \endtable
}
